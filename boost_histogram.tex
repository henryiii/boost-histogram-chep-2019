\documentclass{webofc}
\usepackage[varg]{txfonts}   % Web of Conferences font
\usepackage{hyperref}
%
% Put here some packages required or/and some personnal commands
%
%
\begin{document}
%
\title{Recent developments in histogram libraries}

\author{\firstname{Hans Peter} \lastname{Dembinski}\inst{1} \and
        \firstname{Jim} \lastname{Pivarski}\inst{2} \and
        \firstname{Henry} \lastname{Schreiner}\inst{2}\fnsep\thanks{\email{hschrein@cern.ch}}
        % etc.
}

\institute{
	Max Planck Institute for Nuclear Physics, Heidelberg, Germany
\and
    Princeton University, Princeton, USA 
}

\abstract{%
Boost.Histogram, a header-only C++14 library that provides multi-dimensional histograms and profiles, is now available in Boost-1.70. It is extensible, fast, and uses modern C++ features. Using template meta-programming, the most efficient code path for any given configuration is automatically selected. The library includes key features designed for the particle physics community, such as optional under- and overflow bins, weighted increments, reductions, growing axes, thread-safe filling, and memory-efficient counters with high-dynamic range.

Python bindings for Boost.Histogram are being developed in the Scikit-HEP project to provide a fast, easy-to-install package as a backend for other Python libraries and for advanced users to manipulate histograms. Versatile and efficient histogram filling, effective manipulation, multithreading support, and other features make this a powerful tool. This library has also driven package distribution efforts in Scikit-HEP, allowing binary packages hosted on PyPI to be available for a very wide variety of platforms.

Two other libraries fill out the remainder of the Scikit-HEP Python histogramming effort. Aghast is a library designed to provide conversions between different forms of histograms, enabling interaction between histogram libraries, often without an extra copy in memory. This enables a user to make a histogram in one library and then save it in another form, such as saving a Boost.Histogram in ROOT. And Hist is a library providing friendly, analyst-targeted syntax and shortcuts for quick manipulations and fast plotting using these two libraries.
}

\maketitle

\section{Introduction}
\label{intro}

There is no shortage of histogramming libraries for Python (see Table 1). However, many/most of these are abandoned, have a narrow focus, and most importantly, have little or no interaction with other histogramming libraries. For the Scikit-HEP family 


% This table may move to a webpage somewhere instead.

For a collection of some of the histogram libraries currently available in Python see Table~\ref{tab-1}.

\begin{table}
	\centering
	\caption{Histogram libraries for Python. For the PyPI column, "Pure" means the library is pure Python, "Wheels" means it is compiled but binary wheels are available, "Source" means the code is there, but must be compiled and may require other dependencies, and "No" means it is not hosted on PyPI. Old projects that predate universal (pure python) wheels may be incorrectly listed as "Source".}
	\label{tab-libraries}       % Give a unique label
	\begin{tabular}{llll}
		\hline
		Library                                                                      & Updated & PyPI   & Notes                                                \\ \hline
		\href{https://www.numpy.org/}{numpy}                                         & 2019         & Wheels & Very simple histogramming functions                  \\
		\href{https://coffeateam.github.io/coffea/notebooks/histograms.html}{coffea} & 2019         & Pure   & Family of tools for HEP Columnar analysis            \\
		\href{https://histogrammar.org}{Histogrammar}                                & 2019         & Pure   & Multilanguage, limited support                       \\
		\href{https://pygram11.readthedocs.io}{pygram11}                             & 2019         & Wheels & Unix only, Python 3 only                             \\
		\href{https://root.cern.ch/pyroot}{PyROOT}                                   & 2019         & No     & CERN's ROOT, UNIX binaries on conda-forge            \\
		\href{https://yoda.hepforge.org}{YODA}                                       & 2019         & No     & HEP tool for MCnet                                   \\
		\href{https://physt.readthedocs.io/en/latest/tutorial.html}{physt}           & 2019         & Pure   & Non-HEP specific tool                                \\
		\href{https://github.com/astrofrog/fast-histogram}{fast-histogram}           & 2019         & Yes    & Fast but limited                                     \\
		\href{https://vaex.io}{Vaex}                                                 & 2019         & Source & Large system for data analysis                       \\
		\href{https://pypi.org/project/hdrhistogram/}{hdrhistogram}                  & 2019         & Source & Multilanguage, large range                           \\
		\href{https://pypi.org/project/multihist/}{multihist}                        & 2019         & Pure   & Numpy wrapper for syntax                             \\
		\href{https://github.com/scikit-hep/histbook}{HistBook}                      & 2018         & Pure   & Archived, Replaced by boost-histogram / hist         \\
		\href{https://pypi.org/project/qhist/}{qhist}                                & 2018         & Source & ROOT required, Python 2.7 only                       \\
		\href{https://github.com/theodoregoetz/histogram}{theodoregoetz}             & 2018         & No     & Tried to combine many of the above                   \\
		\href{https://github.com/drdavis/rootplotlib}{rootplotlib}                   & 2016         & No     & ROOT backend                                         \\
		\href{https://pypi.org/project/matplotlib-hep/}{matplotlib-hep}              & 2016         & Source & Focused on plotting                                  \\
		\href{https://github.com/jpivarski/svgfig}{SVGFig}                           & 2016         & No     & Plotting framework                                   \\
		\href{https://github.com/jpivarski/plothon}{Plothon}                         & 2015         & No     & Predecessor to SVGFig                                \\
		\href{https://pypi.org/project/pyhistogram/}{pyhistogram}                    & 2014         & Pure   & Inspired by rootpy                                   \\
		\href{https://pypi.org/project/pypeaks}{pypeaks}                             & 2014         & Pure   & Peak detection                                       \\
		\href{https://github.com/opendatagroup/cassius}{Cassius}                     & 2013         & No     & Statistical Modeling Package                         \\
		\href{https://pypi.org/project/histogramy}{histogramy}                       & 2013         & Pure   & 1D with some fitting tools                           \\
		\href{https://pypi.org/project/histogram}{histogram}                         & 2011         & Source & For Distributed Data Analysis for Neutron Scattering \\
		\href{https://pypi.org/project/SimpleHist/}{SimpleHist}                      & 2011         & Pure   & Numpy based                                          \\
		\href{https://pypi.org/project/paida/}{paida}                                & 2007         & Source & Analysis and plotting                                \\ \hline
	\end{tabular}
\end{table}

\section{Boost.Histogram for C++}
\label{sec-bh-cpp}

% Ideas:
% 
% * Basic overview
% * Include everything mentioned in abstract if possible: optional under- and overflow bins, weighted increments, reductions, growing axes, thread-safe filling, and memory-efficient counters with high-dynamic range.
% * Highlight dynamic features that enable Python (or other) bindings (handy place to highlight compile time benefits when dynamic axes, etc not needed)
% * Describe 1.72's high performance .fill
% * Include performance plot from slides, diagram from slides (I can add images to repo later - Henry)


\section{boost-histogram for Python}
\label{sec-bh-py}


% Hans, please adjust/expand this paragraph as needed! - Henry
Boost.Histogram was developed with Python in mind. Original prototype bindings using Boost.Python were included in the original draft submitted to Boost; however, to keep the library focused they were removed before the first release. New bindings based on PyBind11 were developed as part of the Scikit-HEP family of Python packages.

The new bindings were designed to be ...

\subsection{Design}
\subsection{Flexibility}
\subsection{Speed}
\subsection{Distribution}

\section{The Scikit-HEP family}
\label{sec-3}

\subsection{Hist}
\subsection{Aghast}
\subsection{The Scikit-HEP family}

Don't forget to give each section, subsection, subsubsection, and
paragraph a unique label (see Sect.~\ref{sec-bh-cpp}).

\section{Summary}
\label{sec-4}

% BibTeX or Biber users please use (the style is already called in the class, ensure that the "woc.bst" style is in your local directory)
% \bibliography{name or your bibliography database}


\end{document}

